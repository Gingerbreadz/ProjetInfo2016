% Generated by Sphinx.
\def\sphinxdocclass{report}
\documentclass[letterpaper,10pt,english]{sphinxmanual}
\usepackage[utf8]{inputenc}
\DeclareUnicodeCharacter{00A0}{\nobreakspace}
\usepackage{cmap}
\usepackage[T1]{fontenc}

\usepackage{babel}
\usepackage{times}
\usepackage[Sonny]{fncychap}
\usepackage{longtable}
\usepackage{sphinx}
\usepackage{multirow}
\usepackage{eqparbox}


\addto\captionsenglish{\renewcommand{\figurename}{Fig. }}
\addto\captionsenglish{\renewcommand{\tablename}{Table }}
\SetupFloatingEnvironment{literal-block}{name=Listing }



\title{loganalyser Documentation}
\date{May 24, 2016}
\release{}
\author{Author}
\newcommand{\sphinxlogo}{}
\renewcommand{\releasename}{Release}
\setcounter{tocdepth}{0}
\makeindex

\makeatletter
\def\PYG@reset{\let\PYG@it=\relax \let\PYG@bf=\relax%
    \let\PYG@ul=\relax \let\PYG@tc=\relax%
    \let\PYG@bc=\relax \let\PYG@ff=\relax}
\def\PYG@tok#1{\csname PYG@tok@#1\endcsname}
\def\PYG@toks#1+{\ifx\relax#1\empty\else%
    \PYG@tok{#1}\expandafter\PYG@toks\fi}
\def\PYG@do#1{\PYG@bc{\PYG@tc{\PYG@ul{%
    \PYG@it{\PYG@bf{\PYG@ff{#1}}}}}}}
\def\PYG#1#2{\PYG@reset\PYG@toks#1+\relax+\PYG@do{#2}}

\expandafter\def\csname PYG@tok@s\endcsname{\def\PYG@tc##1{\textcolor[rgb]{0.25,0.44,0.63}{##1}}}
\expandafter\def\csname PYG@tok@nt\endcsname{\let\PYG@bf=\textbf\def\PYG@tc##1{\textcolor[rgb]{0.02,0.16,0.45}{##1}}}
\expandafter\def\csname PYG@tok@mf\endcsname{\def\PYG@tc##1{\textcolor[rgb]{0.13,0.50,0.31}{##1}}}
\expandafter\def\csname PYG@tok@ge\endcsname{\let\PYG@it=\textit}
\expandafter\def\csname PYG@tok@k\endcsname{\let\PYG@bf=\textbf\def\PYG@tc##1{\textcolor[rgb]{0.00,0.44,0.13}{##1}}}
\expandafter\def\csname PYG@tok@kc\endcsname{\let\PYG@bf=\textbf\def\PYG@tc##1{\textcolor[rgb]{0.00,0.44,0.13}{##1}}}
\expandafter\def\csname PYG@tok@cm\endcsname{\let\PYG@it=\textit\def\PYG@tc##1{\textcolor[rgb]{0.25,0.50,0.56}{##1}}}
\expandafter\def\csname PYG@tok@no\endcsname{\def\PYG@tc##1{\textcolor[rgb]{0.38,0.68,0.84}{##1}}}
\expandafter\def\csname PYG@tok@si\endcsname{\let\PYG@it=\textit\def\PYG@tc##1{\textcolor[rgb]{0.44,0.63,0.82}{##1}}}
\expandafter\def\csname PYG@tok@mo\endcsname{\def\PYG@tc##1{\textcolor[rgb]{0.13,0.50,0.31}{##1}}}
\expandafter\def\csname PYG@tok@ss\endcsname{\def\PYG@tc##1{\textcolor[rgb]{0.32,0.47,0.09}{##1}}}
\expandafter\def\csname PYG@tok@il\endcsname{\def\PYG@tc##1{\textcolor[rgb]{0.13,0.50,0.31}{##1}}}
\expandafter\def\csname PYG@tok@sc\endcsname{\def\PYG@tc##1{\textcolor[rgb]{0.25,0.44,0.63}{##1}}}
\expandafter\def\csname PYG@tok@nb\endcsname{\def\PYG@tc##1{\textcolor[rgb]{0.00,0.44,0.13}{##1}}}
\expandafter\def\csname PYG@tok@ni\endcsname{\let\PYG@bf=\textbf\def\PYG@tc##1{\textcolor[rgb]{0.84,0.33,0.22}{##1}}}
\expandafter\def\csname PYG@tok@sx\endcsname{\def\PYG@tc##1{\textcolor[rgb]{0.78,0.36,0.04}{##1}}}
\expandafter\def\csname PYG@tok@m\endcsname{\def\PYG@tc##1{\textcolor[rgb]{0.13,0.50,0.31}{##1}}}
\expandafter\def\csname PYG@tok@mh\endcsname{\def\PYG@tc##1{\textcolor[rgb]{0.13,0.50,0.31}{##1}}}
\expandafter\def\csname PYG@tok@sr\endcsname{\def\PYG@tc##1{\textcolor[rgb]{0.14,0.33,0.53}{##1}}}
\expandafter\def\csname PYG@tok@sd\endcsname{\let\PYG@it=\textit\def\PYG@tc##1{\textcolor[rgb]{0.25,0.44,0.63}{##1}}}
\expandafter\def\csname PYG@tok@gp\endcsname{\let\PYG@bf=\textbf\def\PYG@tc##1{\textcolor[rgb]{0.78,0.36,0.04}{##1}}}
\expandafter\def\csname PYG@tok@cpf\endcsname{\let\PYG@it=\textit\def\PYG@tc##1{\textcolor[rgb]{0.25,0.50,0.56}{##1}}}
\expandafter\def\csname PYG@tok@mb\endcsname{\def\PYG@tc##1{\textcolor[rgb]{0.13,0.50,0.31}{##1}}}
\expandafter\def\csname PYG@tok@err\endcsname{\def\PYG@bc##1{\setlength{\fboxsep}{0pt}\fcolorbox[rgb]{1.00,0.00,0.00}{1,1,1}{\strut ##1}}}
\expandafter\def\csname PYG@tok@cs\endcsname{\def\PYG@tc##1{\textcolor[rgb]{0.25,0.50,0.56}{##1}}\def\PYG@bc##1{\setlength{\fboxsep}{0pt}\colorbox[rgb]{1.00,0.94,0.94}{\strut ##1}}}
\expandafter\def\csname PYG@tok@s1\endcsname{\def\PYG@tc##1{\textcolor[rgb]{0.25,0.44,0.63}{##1}}}
\expandafter\def\csname PYG@tok@nl\endcsname{\let\PYG@bf=\textbf\def\PYG@tc##1{\textcolor[rgb]{0.00,0.13,0.44}{##1}}}
\expandafter\def\csname PYG@tok@nd\endcsname{\let\PYG@bf=\textbf\def\PYG@tc##1{\textcolor[rgb]{0.33,0.33,0.33}{##1}}}
\expandafter\def\csname PYG@tok@vg\endcsname{\def\PYG@tc##1{\textcolor[rgb]{0.73,0.38,0.84}{##1}}}
\expandafter\def\csname PYG@tok@nn\endcsname{\let\PYG@bf=\textbf\def\PYG@tc##1{\textcolor[rgb]{0.05,0.52,0.71}{##1}}}
\expandafter\def\csname PYG@tok@gh\endcsname{\let\PYG@bf=\textbf\def\PYG@tc##1{\textcolor[rgb]{0.00,0.00,0.50}{##1}}}
\expandafter\def\csname PYG@tok@gu\endcsname{\let\PYG@bf=\textbf\def\PYG@tc##1{\textcolor[rgb]{0.50,0.00,0.50}{##1}}}
\expandafter\def\csname PYG@tok@sh\endcsname{\def\PYG@tc##1{\textcolor[rgb]{0.25,0.44,0.63}{##1}}}
\expandafter\def\csname PYG@tok@bp\endcsname{\def\PYG@tc##1{\textcolor[rgb]{0.00,0.44,0.13}{##1}}}
\expandafter\def\csname PYG@tok@gr\endcsname{\def\PYG@tc##1{\textcolor[rgb]{1.00,0.00,0.00}{##1}}}
\expandafter\def\csname PYG@tok@w\endcsname{\def\PYG@tc##1{\textcolor[rgb]{0.73,0.73,0.73}{##1}}}
\expandafter\def\csname PYG@tok@sb\endcsname{\def\PYG@tc##1{\textcolor[rgb]{0.25,0.44,0.63}{##1}}}
\expandafter\def\csname PYG@tok@se\endcsname{\let\PYG@bf=\textbf\def\PYG@tc##1{\textcolor[rgb]{0.25,0.44,0.63}{##1}}}
\expandafter\def\csname PYG@tok@kn\endcsname{\let\PYG@bf=\textbf\def\PYG@tc##1{\textcolor[rgb]{0.00,0.44,0.13}{##1}}}
\expandafter\def\csname PYG@tok@vc\endcsname{\def\PYG@tc##1{\textcolor[rgb]{0.73,0.38,0.84}{##1}}}
\expandafter\def\csname PYG@tok@vi\endcsname{\def\PYG@tc##1{\textcolor[rgb]{0.73,0.38,0.84}{##1}}}
\expandafter\def\csname PYG@tok@ne\endcsname{\def\PYG@tc##1{\textcolor[rgb]{0.00,0.44,0.13}{##1}}}
\expandafter\def\csname PYG@tok@kp\endcsname{\def\PYG@tc##1{\textcolor[rgb]{0.00,0.44,0.13}{##1}}}
\expandafter\def\csname PYG@tok@s2\endcsname{\def\PYG@tc##1{\textcolor[rgb]{0.25,0.44,0.63}{##1}}}
\expandafter\def\csname PYG@tok@nf\endcsname{\def\PYG@tc##1{\textcolor[rgb]{0.02,0.16,0.49}{##1}}}
\expandafter\def\csname PYG@tok@na\endcsname{\def\PYG@tc##1{\textcolor[rgb]{0.25,0.44,0.63}{##1}}}
\expandafter\def\csname PYG@tok@gs\endcsname{\let\PYG@bf=\textbf}
\expandafter\def\csname PYG@tok@gi\endcsname{\def\PYG@tc##1{\textcolor[rgb]{0.00,0.63,0.00}{##1}}}
\expandafter\def\csname PYG@tok@kr\endcsname{\let\PYG@bf=\textbf\def\PYG@tc##1{\textcolor[rgb]{0.00,0.44,0.13}{##1}}}
\expandafter\def\csname PYG@tok@gt\endcsname{\def\PYG@tc##1{\textcolor[rgb]{0.00,0.27,0.87}{##1}}}
\expandafter\def\csname PYG@tok@gd\endcsname{\def\PYG@tc##1{\textcolor[rgb]{0.63,0.00,0.00}{##1}}}
\expandafter\def\csname PYG@tok@ow\endcsname{\let\PYG@bf=\textbf\def\PYG@tc##1{\textcolor[rgb]{0.00,0.44,0.13}{##1}}}
\expandafter\def\csname PYG@tok@ch\endcsname{\let\PYG@it=\textit\def\PYG@tc##1{\textcolor[rgb]{0.25,0.50,0.56}{##1}}}
\expandafter\def\csname PYG@tok@kt\endcsname{\def\PYG@tc##1{\textcolor[rgb]{0.56,0.13,0.00}{##1}}}
\expandafter\def\csname PYG@tok@cp\endcsname{\def\PYG@tc##1{\textcolor[rgb]{0.00,0.44,0.13}{##1}}}
\expandafter\def\csname PYG@tok@c1\endcsname{\let\PYG@it=\textit\def\PYG@tc##1{\textcolor[rgb]{0.25,0.50,0.56}{##1}}}
\expandafter\def\csname PYG@tok@go\endcsname{\def\PYG@tc##1{\textcolor[rgb]{0.20,0.20,0.20}{##1}}}
\expandafter\def\csname PYG@tok@nv\endcsname{\def\PYG@tc##1{\textcolor[rgb]{0.73,0.38,0.84}{##1}}}
\expandafter\def\csname PYG@tok@o\endcsname{\def\PYG@tc##1{\textcolor[rgb]{0.40,0.40,0.40}{##1}}}
\expandafter\def\csname PYG@tok@kd\endcsname{\let\PYG@bf=\textbf\def\PYG@tc##1{\textcolor[rgb]{0.00,0.44,0.13}{##1}}}
\expandafter\def\csname PYG@tok@c\endcsname{\let\PYG@it=\textit\def\PYG@tc##1{\textcolor[rgb]{0.25,0.50,0.56}{##1}}}
\expandafter\def\csname PYG@tok@nc\endcsname{\let\PYG@bf=\textbf\def\PYG@tc##1{\textcolor[rgb]{0.05,0.52,0.71}{##1}}}
\expandafter\def\csname PYG@tok@mi\endcsname{\def\PYG@tc##1{\textcolor[rgb]{0.13,0.50,0.31}{##1}}}

\def\PYGZbs{\char`\\}
\def\PYGZus{\char`\_}
\def\PYGZob{\char`\{}
\def\PYGZcb{\char`\}}
\def\PYGZca{\char`\^}
\def\PYGZam{\char`\&}
\def\PYGZlt{\char`\<}
\def\PYGZgt{\char`\>}
\def\PYGZsh{\char`\#}
\def\PYGZpc{\char`\%}
\def\PYGZdl{\char`\$}
\def\PYGZhy{\char`\-}
\def\PYGZsq{\char`\'}
\def\PYGZdq{\char`\"}
\def\PYGZti{\char`\~}
% for compatibility with earlier versions
\def\PYGZat{@}
\def\PYGZlb{[}
\def\PYGZrb{]}
\makeatother

\renewcommand\PYGZsq{\textquotesingle}

\begin{document}

\maketitle
\tableofcontents
\phantomsection\label{index::doc}



\chapter{Description}
\label{index:welcome-to-loganalyser-s-documentation}\label{index:description}
loganalyser est un analyseur de journal d'activité de serveur web Apache, réalisé dans le cadre du projet
développement informatique en première année du cursus ingénieur de Télécom SudParis.


\chapter{Table des matières}
\label{index:table-des-matieres}

\section{Installation}
\label{installation::doc}\label{installation:installation}
Le projet reprend la structure standard d'un package Python, et peut donc facilement être installé via le \code{setup.py}.
Conçu pour \code{python3.X}, il convient avant de procéder à l'installation de s'assurer de la version de l'environnement
Python courant.


\subsection{Installation directe}
\label{installation:installation-directe}
Cette procedure installera le package au sein de Python comme module. Le projet a été conçu pour que les étapes à suivre
soient les même que pour tout autre module.

Il est possible de se placer, avant d'initier la procedure, dans un environnement Python virtuelle \code{venv} si
l'environnement courant n'est pas approprié.

\begin{Verbatim}[commandchars=\\\{\}]
\PYGZdl{} git clone https://github.com/Gingerbreadz/ProjetInfo2016
\PYGZdl{} \PYG{n+nb}{cd} ./ProjetInfo2016
\PYGZdl{} python setup.py install
\end{Verbatim}


\subsection{Installation dans pip}
\label{installation:installation-dans-pip}
Une procédure alternative permet d'installer le projet comme package \code{pip} et ainsi de plus facilement le désinstaller.
Le projet n'étant pas disponible sur les dépots PyPi, l'installation se déroule comme suit.

\begin{Verbatim}[commandchars=\\\{\}]
\PYGZdl{} git clone https://github.com/Gingerbreadz/ProjetInfo2016
\PYGZdl{} \PYG{n+nb}{cd} ./ProjetInfo2016
\PYGZdl{} python setup.py sdist
\PYGZdl{} pip install ./dist/loganalyser\PYGZhy{}0.0.1.tar.gz
\end{Verbatim}

La désinstallation peut alors être faite avec un \code{pip uninstall} comme pour tout autre package pip. En cas de
difficultés à identifier le nom du package, \code{pip freeze} permet de lister tout les package pip installés dans
l'environnement Python courant.


\section{Usage}
\label{usage::doc}\label{usage:usage}
L'utilisation du programme nécessite la possession du fichier de log que l'on souhaite analyser.
Une fois le paquet installé, se fait directement depuis la ligne de commande de la façon :
\begin{quote}

\begin{Verbatim}[commandchars=\\\{\}]
\PYGZdl{} loganalyser f \PYG{o}{[}n\PYG{o}{]} \PYG{o}{[}o\PYG{o}{]}
\end{Verbatim}
\end{quote}

\textbf{Paramètres}
\begin{itemize}
\item {} 
f : Chemin fichier de logs.

\item {} 
n : (Optionel) Nombre de ligne à afficher

\item {} 
o : (Optionel) Fichier de sortie.

\end{itemize}

\textbf{Exemple}
\begin{quote}

\begin{Verbatim}[commandchars=\\\{\}]
\PYGZdl{} loganalyser \PYG{l+s+s2}{\PYGZdq{}/chemin/vers/fichier/de/log\PYGZdq{}} \PYG{l+m}{6} \PYG{l+s+s2}{\PYGZdq{}/chemin/fichier/output\PYGZdq{}}
\end{Verbatim}
\end{quote}


\section{Tests}
\label{tests::doc}\label{tests:tests}
Les tests sont disponible directement sous \code{/tests/} et executable comme un simple script Python une fois le programme
installé. Les tests utilisent des fichiers de log disponibles sous \code{/res/}. Pour exectuer un test depuis le
dossier parent du projet :
\begin{quote}

\begin{Verbatim}[commandchars=\\\{\}]
\PYGZdl{} python ./tests/fichierdetest.py
\end{Verbatim}
\end{quote}


\section{Loganalyser package}
\label{loganalyser::doc}\label{loganalyser:loganalyser-package}

\subsection{Diagnostique module}
\label{loganalyser:diagnostique-module}\label{loganalyser:module-loganalyser.diagnostique}\index{loganalyser.diagnostique (module)}
Implementation du Diagnostique.
\index{Diagnostique (class in loganalyser.diagnostique)}

\begin{fulllineitems}
\phantomsection\label{loganalyser:loganalyser.diagnostique.Diagnostique}\pysiglinewithargsret{\strong{class }\code{loganalyser.diagnostique.}\bfcode{Diagnostique}}{\emph{token\_dictionary}, \emph{regexp\_dictionary}, \emph{n=5}, \emph{nomatchcount=0}}{}
Bases: \code{object}

Classe instanciant le diagnostique, qui contient les résultats d'analyse et les fait.
\index{\_\_init\_\_() (loganalyser.diagnostique.Diagnostique method)}

\begin{fulllineitems}
\phantomsection\label{loganalyser:loganalyser.diagnostique.Diagnostique.__init__}\pysiglinewithargsret{\bfcode{\_\_init\_\_}}{\emph{token\_dictionary}, \emph{regexp\_dictionary}, \emph{n=5}, \emph{nomatchcount=0}}{}
Constructeur de classe. Un diagnostique est initialisé à partir de tokens.
\begin{quote}\begin{description}
\item[{Parameters}] \leavevmode\begin{itemize}
\item {} 
\textbf{\texttt{token\_dictionary}} (\emph{\texttt{dict}}) -- dictionnaire de token

\item {} 
\textbf{\texttt{regexp\_dictionary}} (\emph{\texttt{dict}}) -- dictionnaire de regexp

\item {} 
\textbf{\texttt{n}} (\emph{\texttt{int}}) -- nombre de ligne à afficher (5 par défaut)

\item {} 
\textbf{\texttt{nomatchcount}} (\emph{\texttt{int}}) -- nombre de ligne n'ayant pas matchés (0 par défaut)

\end{itemize}

\end{description}\end{quote}

\end{fulllineitems}

\index{get\_indices\_top() (loganalyser.diagnostique.Diagnostique method)}

\begin{fulllineitems}
\phantomsection\label{loganalyser:loganalyser.diagnostique.Diagnostique.get_indices_top}\pysiglinewithargsret{\bfcode{get\_indices\_top}}{\emph{liste}}{}
Permet de trier une liste nous permettant de récupérer des valeurs triées de nos données à l'affichage.
\begin{quote}\begin{description}
\item[{Parameters}] \leavevmode
\textbf{\texttt{liste}} (\emph{\texttt{list}}) -- une liste

\item[{Returns}] \leavevmode
indices des valeurs que l'on souhaite afficher dans l'ordre de leurs futur affichage

\item[{Return type}] \leavevmode
list

\end{description}\end{quote}

\end{fulllineitems}

\index{get\_topfiles() (loganalyser.diagnostique.Diagnostique method)}

\begin{fulllineitems}
\phantomsection\label{loganalyser:loganalyser.diagnostique.Diagnostique.get_topfiles}\pysiglinewithargsret{\bfcode{get\_topfiles}}{\emph{stat}}{}
Ordonne la liste des top files et s'assure de sa configuration afin d'obtenir un affichage lisible des
résultats.
\begin{quote}\begin{description}
\item[{Parameters}] \leavevmode
\textbf{\texttt{stat}} (\emph{\texttt{dict}}) -- dictionnaire de statistiques

\item[{Returns}] \leavevmode
Liste des strings organisées.

\item[{Return type}] \leavevmode
list

\end{description}\end{quote}

\end{fulllineitems}

\index{get\_topreferrers() (loganalyser.diagnostique.Diagnostique method)}

\begin{fulllineitems}
\phantomsection\label{loganalyser:loganalyser.diagnostique.Diagnostique.get_topreferrers}\pysiglinewithargsret{\bfcode{get\_topreferrers}}{\emph{stat}}{}
Ordonne la liste des top referrers et s'assure de sa configuration afin d'obtenir un affichage lisible des
résultats.
\begin{quote}\begin{description}
\item[{Parameters}] \leavevmode
\textbf{\texttt{stat}} (\emph{\texttt{dict}}) -- dictionnaire de statistiques

\item[{Returns}] \leavevmode
Liste des strings organisées.

\item[{Return type}] \leavevmode
list

\end{description}\end{quote}

\end{fulllineitems}

\index{get\_topvisitors() (loganalyser.diagnostique.Diagnostique method)}

\begin{fulllineitems}
\phantomsection\label{loganalyser:loganalyser.diagnostique.Diagnostique.get_topvisitors}\pysiglinewithargsret{\bfcode{get\_topvisitors}}{\emph{stat}}{}
Ordonne la liste des top visitors et s'assure de sa configuration afin d'obtenir un affichage lisible des
résultats
\begin{quote}\begin{description}
\item[{Parameters}] \leavevmode
\textbf{\texttt{stat}} (\emph{\texttt{dict}}) -- dictionnaire de statistiques

\item[{Returns}] \leavevmode
Liste des strings organisées.

\item[{Return type}] \leavevmode
list

\end{description}\end{quote}

\end{fulllineitems}

\index{get\_topuniqueresponses() (loganalyser.diagnostique.Diagnostique method)}

\begin{fulllineitems}
\phantomsection\label{loganalyser:loganalyser.diagnostique.Diagnostique.get_topuniqueresponses}\pysiglinewithargsret{\bfcode{get\_topuniqueresponses}}{\emph{stat}}{}
Ordonne la liste des top unique responses et s'assure de sa configuration afin d'obtenir un affichage lisible
des résultats.
\begin{quote}\begin{description}
\item[{Parameters}] \leavevmode
\textbf{\texttt{stat}} (\emph{\texttt{dict}}) -- dictionnaire de statistiques

\item[{Returns}] \leavevmode
Liste des strings organisées.

\item[{Return type}] \leavevmode
list

\end{description}\end{quote}

\end{fulllineitems}

\index{get\_attack() (loganalyser.diagnostique.Diagnostique method)}

\begin{fulllineitems}
\phantomsection\label{loganalyser:loganalyser.diagnostique.Diagnostique.get_attack}\pysiglinewithargsret{\bfcode{get\_attack}}{\emph{attack}}{}
Ordonne la liste des potentiels attaques et s'assure de sa configuration afin d'obtenir un affichage lisible
des résultats.
\begin{quote}\begin{description}
\item[{Parameters}] \leavevmode
\textbf{\texttt{attack}} (\emph{\texttt{dict}}) -- dictionnaire d'attaque.

\item[{Returns}] \leavevmode
Liste des strings organisées.

\item[{Return type}] \leavevmode
list

\end{description}\end{quote}

\end{fulllineitems}

\index{get\_report() (loganalyser.diagnostique.Diagnostique method)}

\begin{fulllineitems}
\phantomsection\label{loganalyser:loganalyser.diagnostique.Diagnostique.get_report}\pysiglinewithargsret{\bfcode{get\_report}}{}{}
Ordonne les donnée issues des statistiques et des analyses, prépare pour l'affichage finale.
\begin{quote}\begin{description}
\item[{Returns}] \leavevmode
tableau des lignes de résultats à partir des dictionnaires

\item[{Return type}] \leavevmode
list

\end{description}\end{quote}

\end{fulllineitems}

\index{\_Diagnostique\_\_analyse() (loganalyser.diagnostique.Diagnostique method)}

\begin{fulllineitems}
\phantomsection\label{loganalyser:loganalyser.diagnostique.Diagnostique._Diagnostique__analyse}\pysiglinewithargsret{\bfcode{\_Diagnostique\_\_analyse}}{}{}
Analyse les tokens par groupe selon certains motifs.
\begin{quote}\begin{description}
\item[{Parameters}] \leavevmode\begin{itemize}
\item {} 
\textbf{\texttt{self.token\_dict}} (\emph{\texttt{dict}}) -- dictionnaire de token

\item {} 
\textbf{\texttt{self.regexp\_dict}} (\emph{\texttt{dict}}) -- dictionnaire d'expression régulière

\end{itemize}

\item[{Returns}] \leavevmode
Dictionnaire contenant le rapport des attaques subit

\item[{Return type}] \leavevmode
dict

\end{description}\end{quote}

\end{fulllineitems}

\index{\_Diagnostique\_\_statistique() (loganalyser.diagnostique.Diagnostique method)}

\begin{fulllineitems}
\phantomsection\label{loganalyser:loganalyser.diagnostique.Diagnostique._Diagnostique__statistique}\pysiglinewithargsret{\bfcode{\_Diagnostique\_\_statistique}}{}{}
Effectue des calculs statistiques sur les token.
\begin{quote}\begin{description}
\item[{Parameters}] \leavevmode
\textbf{\texttt{self.token\_dict}} (\emph{\texttt{dict}}) -- dictionnaire de token

\item[{Returns}] \leavevmode
Dictionnaire contenant les statistiques

\item[{Return type}] \leavevmode
dict

\end{description}\end{quote}

\end{fulllineitems}


\end{fulllineitems}



\subsection{Fichier module}
\label{loganalyser:fichier-module}\label{loganalyser:module-loganalyser.fichier}\index{loganalyser.fichier (module)}
Sert à intéragir avec les fichiers.
\index{Fichier (class in loganalyser.fichier)}

\begin{fulllineitems}
\phantomsection\label{loganalyser:loganalyser.fichier.Fichier}\pysiglinewithargsret{\strong{class }\code{loganalyser.fichier.}\bfcode{Fichier}}{\emph{filepath}}{}
Bases: \code{object}

Classe abstraite interface pour fichier caractérisé par :
\begin{itemize}
\item {} 
son nombre de ligne

\item {} 
son contenu

\item {} 
son chemin d'accès

\item {} 
si il est read-only ou non

\end{itemize}
\index{\_\_init\_\_() (loganalyser.fichier.Fichier method)}

\begin{fulllineitems}
\phantomsection\label{loganalyser:loganalyser.fichier.Fichier.__init__}\pysiglinewithargsret{\bfcode{\_\_init\_\_}}{\emph{filepath}}{}
Constructeur de classe. Un fichier est initialisé à partir de son chemin d'accès
\begin{quote}\begin{description}
\item[{Parameters}] \leavevmode
\textbf{\texttt{filepath}} (\emph{\texttt{str}}) -- chemin d'accès du fichier

\end{description}\end{quote}

\end{fulllineitems}

\index{lireligne() (loganalyser.fichier.Fichier method)}

\begin{fulllineitems}
\phantomsection\label{loganalyser:loganalyser.fichier.Fichier.lireligne}\pysiglinewithargsret{\bfcode{lireligne}}{\emph{noligne}}{}
Retourne la ligne n d'un fichier
\begin{quote}\begin{description}
\item[{Parameters}] \leavevmode
\textbf{\texttt{noligne}} (\emph{\texttt{int}}) -- numero de la ligne voulu

\item[{Returns}] \leavevmode
ligne n du fichier instancié

\item[{Return type}] \leavevmode
str

\end{description}\end{quote}

\end{fulllineitems}

\index{fermerfichier() (loganalyser.fichier.Fichier method)}

\begin{fulllineitems}
\phantomsection\label{loganalyser:loganalyser.fichier.Fichier.fermerfichier}\pysiglinewithargsret{\bfcode{fermerfichier}}{}{}
Ferme le fichier pour libérer des ressources

\end{fulllineitems}


\end{fulllineitems}

\index{FichierDeLog (class in loganalyser.fichier)}

\begin{fulllineitems}
\phantomsection\label{loganalyser:loganalyser.fichier.FichierDeLog}\pysiglinewithargsret{\strong{class }\code{loganalyser.fichier.}\bfcode{FichierDeLog}}{\emph{filepath}}{}
Bases: {\hyperref[loganalyser:loganalyser.fichier.Fichier]{\emph{\code{loganalyser.fichier.Fichier}}}}

Classe instanciant des fichiers de log caractérisé par :
\begin{itemize}
\item {} 
son nombre de ligne

\item {} 
son contenu

\item {} 
son chemin d'accès

\item {} 
si il est read-only ou non

\end{itemize}
\index{decouperligne() (loganalyser.fichier.FichierDeLog method)}

\begin{fulllineitems}
\phantomsection\label{loganalyser:loganalyser.fichier.FichierDeLog.decouperligne}\pysiglinewithargsret{\bfcode{decouperligne}}{\emph{noligne}}{}
Decoupage syntaxique de la n-ieme ligne pour séparer les différents token
\begin{quote}\begin{description}
\item[{Parameters}] \leavevmode
\textbf{\texttt{noligne}} (\emph{\texttt{int}}) -- Numéro de ligne

\item[{Returns}] \leavevmode
Liste contenant les différents champs découpés.

\item[{Return type}] \leavevmode
list

\end{description}\end{quote}

\end{fulllineitems}


\end{fulllineitems}

\index{FichierRegExp (class in loganalyser.fichier)}

\begin{fulllineitems}
\phantomsection\label{loganalyser:loganalyser.fichier.FichierRegExp}\pysiglinewithargsret{\strong{class }\code{loganalyser.fichier.}\bfcode{FichierRegExp}}{\emph{filepath}}{}
Bases: {\hyperref[loganalyser:loganalyser.fichier.Fichier]{\emph{\code{loganalyser.fichier.Fichier}}}}

Classe instanciant des fichiers d'expressions régulières caractérisé par :
\begin{itemize}
\item {} 
son nombre de ligne

\item {} 
son contenu

\item {} 
son chemin d'accès

\item {} 
si il est read-only ou non

\end{itemize}
\index{decouperligne() (loganalyser.fichier.FichierRegExp method)}

\begin{fulllineitems}
\phantomsection\label{loganalyser:loganalyser.fichier.FichierRegExp.decouperligne}\pysiglinewithargsret{\bfcode{decouperligne}}{\emph{noligne}}{}
Decoupage syntaxique de la n-ieme ligne pour récupérer les regExp
\begin{quote}\begin{description}
\item[{Parameters}] \leavevmode
\textbf{\texttt{noligne}} (\emph{\texttt{int}}) -- Numéro de ligne

\item[{Returns}] \leavevmode
Liste contenant les différents champs découpés.

\item[{Return type}] \leavevmode
list

\end{description}\end{quote}

\end{fulllineitems}


\end{fulllineitems}

\index{FichierRapportTextuel (class in loganalyser.fichier)}

\begin{fulllineitems}
\phantomsection\label{loganalyser:loganalyser.fichier.FichierRapportTextuel}\pysiglinewithargsret{\strong{class }\code{loganalyser.fichier.}\bfcode{FichierRapportTextuel}}{\emph{filepath}}{}
Bases: {\hyperref[loganalyser:loganalyser.fichier.Fichier]{\emph{\code{loganalyser.fichier.Fichier}}}}

Classe instanciant le rapport textuel caractérisé par :
\begin{itemize}
\item {} 
son nombre de ligne

\item {} 
son contenu

\item {} 
son chemin d'accès

\item {} 
si il est read-only ou non

\end{itemize}
\index{\_\_init\_\_() (loganalyser.fichier.FichierRapportTextuel method)}

\begin{fulllineitems}
\phantomsection\label{loganalyser:loganalyser.fichier.FichierRapportTextuel.__init__}\pysiglinewithargsret{\bfcode{\_\_init\_\_}}{\emph{filepath}}{}
Constructeur de classe. Un fichier est initialisé à partir de son chemin d'accès
\begin{quote}\begin{description}
\item[{Parameters}] \leavevmode
\textbf{\texttt{filepath}} (\emph{\texttt{str}}) -- chemin d'accès du fichier

\end{description}\end{quote}

\end{fulllineitems}

\index{ecriretexte() (loganalyser.fichier.FichierRapportTextuel method)}

\begin{fulllineitems}
\phantomsection\label{loganalyser:loganalyser.fichier.FichierRapportTextuel.ecriretexte}\pysiglinewithargsret{\bfcode{ecriretexte}}{\emph{data}}{}
Ecrit les lignes en entrée à la fin du fichier
\begin{quote}\begin{description}
\item[{Parameters}] \leavevmode
\textbf{\texttt{data}} (\emph{\texttt{list}}) -- numero de la ligne voulu

\end{description}\end{quote}

\end{fulllineitems}


\end{fulllineitems}



\subsection{Token module}
\label{loganalyser:module-loganalyser.token}\label{loganalyser:token-module}\index{loganalyser.token (module)}
Module token
Ce sont les classes qui sont utilisées pour caractériser les différents champs de log.
A l'instanciation de chacune des classes correspondant à un champ, la vérification du type de la donnée est effectuée et lève une erreur si le type n'est pas le bon.
\index{Token (class in loganalyser.token)}

\begin{fulllineitems}
\phantomsection\label{loganalyser:loganalyser.token.Token}\pysiglinewithargsret{\strong{class }\code{loganalyser.token.}\bfcode{Token}}{\emph{value}, \emph{istypeok}}{}
Bases: \code{object}

Classe abstraite interface pour token caractérisé par :
- sa donnée
- sa sévérité
\index{\_\_init\_\_() (loganalyser.token.Token method)}

\begin{fulllineitems}
\phantomsection\label{loganalyser:loganalyser.token.Token.__init__}\pysiglinewithargsret{\bfcode{\_\_init\_\_}}{\emph{value}, \emph{istypeok}}{}
Constructeur de classe. Un fichier est initialisé à partir de son chemin d'accès
\begin{quote}\begin{description}
\item[{Parameters}] \leavevmode\begin{itemize}
\item {} 
\textbf{\texttt{value}} (\emph{\texttt{str}}) -- donnee du token e.g. ``127.0.0.1'', ``404''.

\item {} 
\textbf{\texttt{istypeok}} (\emph{\texttt{bool}}) -- booléen rendant autorisant la création du token.

\end{itemize}

\end{description}\end{quote}

\end{fulllineitems}

\index{\_Token\_\_analyse() (loganalyser.token.Token method)}

\begin{fulllineitems}
\phantomsection\label{loganalyser:loganalyser.token.Token._Token__analyse}\pysiglinewithargsret{\bfcode{\_Token\_\_analyse}}{}{}
Analyse la donnee contenue dans le token pour obtenir la sévérité de cette donnee. Non implémenté car non-utile.
\begin{quote}\begin{description}
\item[{Returns}] \leavevmode
Retourne la sévérité de la donnee de ce token

\item[{Return type}] \leavevmode
int

\end{description}\end{quote}

\end{fulllineitems}

\index{\_Token\_\_verifier\_type() (loganalyser.token.Token method)}

\begin{fulllineitems}
\phantomsection\label{loganalyser:loganalyser.token.Token._Token__verifier_type}\pysiglinewithargsret{\bfcode{\_Token\_\_verifier\_type}}{\emph{value}}{}
Vérifie si la donnee peut bien être instanciée sous cette classe de Token.
\begin{quote}\begin{description}
\item[{Parameters}] \leavevmode
\textbf{\texttt{value}} (\emph{\texttt{str}}) -- valeur de création du token

\item[{Returns}] \leavevmode
Retourne la réponse de la vérification

\item[{Return type}] \leavevmode
bool

\end{description}\end{quote}

\end{fulllineitems}


\end{fulllineitems}

\index{IP (class in loganalyser.token)}

\begin{fulllineitems}
\phantomsection\label{loganalyser:loganalyser.token.IP}\pysiglinewithargsret{\strong{class }\code{loganalyser.token.}\bfcode{IP}}{\emph{value}}{}
Bases: {\hyperref[loganalyser:loganalyser.token.Token]{\emph{\code{loganalyser.token.Token}}}}

Classe concrète instanciant les token IP, le format attendu étant une adresse ipv4 ou ipv6
\index{\_Token\_\_analyse() (loganalyser.token.IP method)}

\begin{fulllineitems}
\phantomsection\label{loganalyser:loganalyser.token.IP._Token__analyse}\pysiglinewithargsret{\bfcode{\_Token\_\_analyse}}{}{}
Analyse la donnee contenue dans le token pour obtenir la sévérité de cette donnee. Non implémenté car non-utile.
\begin{quote}\begin{description}
\item[{Returns}] \leavevmode
Retourne la sévérité de la donnee de ce token

\item[{Return type}] \leavevmode
int

\end{description}\end{quote}

\end{fulllineitems}

\index{\_Token\_\_verifier\_type() (loganalyser.token.IP method)}

\begin{fulllineitems}
\phantomsection\label{loganalyser:loganalyser.token.IP._Token__verifier_type}\pysiglinewithargsret{\bfcode{\_Token\_\_verifier\_type}}{\emph{value}}{}
Vérifie si la donnee peut bien être instanciée sous cette classe de Token.
\begin{quote}\begin{description}
\item[{Parameters}] \leavevmode
\textbf{\texttt{value}} (\emph{\texttt{str}}) -- valeur de création du token

\item[{Returns}] \leavevmode
Retourne la réponse de la vérification

\item[{Return type}] \leavevmode
bool

\end{description}\end{quote}

\end{fulllineitems}


\end{fulllineitems}

\index{Name (class in loganalyser.token)}

\begin{fulllineitems}
\phantomsection\label{loganalyser:loganalyser.token.Name}\pysiglinewithargsret{\strong{class }\code{loganalyser.token.}\bfcode{Name}}{\emph{value}}{}
Bases: {\hyperref[loganalyser:loganalyser.token.Token]{\emph{\code{loganalyser.token.Token}}}}

Classe concrète instanciant les token Nom, le format attendu étant une chaine de caractères
\index{\_Token\_\_analyse() (loganalyser.token.Name method)}

\begin{fulllineitems}
\phantomsection\label{loganalyser:loganalyser.token.Name._Token__analyse}\pysiglinewithargsret{\bfcode{\_Token\_\_analyse}}{}{}
Analyse la donnee contenue dans le token pour obtenir la sévérité de cette donnee. Non implémenté car non-utile.
\begin{quote}\begin{description}
\item[{Returns}] \leavevmode
Retourne la sévérité de la donnee de ce token

\item[{Return type}] \leavevmode
int

\end{description}\end{quote}

\end{fulllineitems}

\index{\_Token\_\_verifier\_type() (loganalyser.token.Name method)}

\begin{fulllineitems}
\phantomsection\label{loganalyser:loganalyser.token.Name._Token__verifier_type}\pysiglinewithargsret{\bfcode{\_Token\_\_verifier\_type}}{\emph{value}}{}
Vérifie si la donnee peut bien être instanciée sous cette classe de Token.
\begin{quote}\begin{description}
\item[{Parameters}] \leavevmode
\textbf{\texttt{value}} (\emph{\texttt{str}}) -- valeur de création du token

\item[{Returns}] \leavevmode
Retourne la réponse de la vérification

\item[{Return type}] \leavevmode
bool

\end{description}\end{quote}

\end{fulllineitems}


\end{fulllineitems}

\index{Date (class in loganalyser.token)}

\begin{fulllineitems}
\phantomsection\label{loganalyser:loganalyser.token.Date}\pysiglinewithargsret{\strong{class }\code{loganalyser.token.}\bfcode{Date}}{\emph{value}}{}
Bases: {\hyperref[loganalyser:loganalyser.token.Token]{\emph{\code{loganalyser.token.Token}}}}

Classe concrète instanciant les token Date, le format attendu étant JJ/MM/YYYY:HH:MM:SS
\index{\_Token\_\_analyse() (loganalyser.token.Date method)}

\begin{fulllineitems}
\phantomsection\label{loganalyser:loganalyser.token.Date._Token__analyse}\pysiglinewithargsret{\bfcode{\_Token\_\_analyse}}{}{}
Analyse la donnee contenue dans le token pour obtenir la sévérité de cette donnee. Non implémenté car non-utile.
\begin{quote}\begin{description}
\item[{Returns}] \leavevmode
Retourne la sévérité de la donnee de ce token

\item[{Return type}] \leavevmode
int

\end{description}\end{quote}

\end{fulllineitems}

\index{\_Token\_\_verifier\_type() (loganalyser.token.Date method)}

\begin{fulllineitems}
\phantomsection\label{loganalyser:loganalyser.token.Date._Token__verifier_type}\pysiglinewithargsret{\bfcode{\_Token\_\_verifier\_type}}{\emph{value}}{}
Vérifie si la donnee peut bien être instanciée sous cette classe de Token.
\begin{quote}\begin{description}
\item[{Parameters}] \leavevmode
\textbf{\texttt{value}} (\emph{\texttt{str}}) -- valeur de création du token

\item[{Returns}] \leavevmode
Retourne la réponse de la vérification

\item[{Return type}] \leavevmode
bool

\end{description}\end{quote}

\end{fulllineitems}


\end{fulllineitems}

\index{EXT (class in loganalyser.token)}

\begin{fulllineitems}
\phantomsection\label{loganalyser:loganalyser.token.EXT}\pysiglinewithargsret{\strong{class }\code{loganalyser.token.}\bfcode{EXT}}{\emph{value}}{}
Bases: {\hyperref[loganalyser:loganalyser.token.Token]{\emph{\code{loganalyser.token.Token}}}}

Classe concrète instanciant les token Ext, le format attendu étant un entier
\index{\_Token\_\_analyse() (loganalyser.token.EXT method)}

\begin{fulllineitems}
\phantomsection\label{loganalyser:loganalyser.token.EXT._Token__analyse}\pysiglinewithargsret{\bfcode{\_Token\_\_analyse}}{}{}
Analyse la donnee contenue dans le token pour obtenir la sévérité de cette donnee. Non implémenté car non-utile.
\begin{quote}\begin{description}
\item[{Returns}] \leavevmode
Retourne la sévérité de la donnee de ce token

\item[{Return type}] \leavevmode
int

\end{description}\end{quote}

\end{fulllineitems}

\index{\_Token\_\_verifier\_type() (loganalyser.token.EXT method)}

\begin{fulllineitems}
\phantomsection\label{loganalyser:loganalyser.token.EXT._Token__verifier_type}\pysiglinewithargsret{\bfcode{\_Token\_\_verifier\_type}}{\emph{value}}{}
Vérifie si la donnee peut bien être instanciée sous cette classe de Token.
\begin{quote}\begin{description}
\item[{Parameters}] \leavevmode
\textbf{\texttt{value}} (\emph{\texttt{str}}) -- valeur de création du token

\item[{Returns}] \leavevmode
Retourne la réponse de la vérification

\item[{Return type}] \leavevmode
bool

\end{description}\end{quote}

\end{fulllineitems}


\end{fulllineitems}

\index{Method (class in loganalyser.token)}

\begin{fulllineitems}
\phantomsection\label{loganalyser:loganalyser.token.Method}\pysiglinewithargsret{\strong{class }\code{loganalyser.token.}\bfcode{Method}}{\emph{value}}{}
Bases: {\hyperref[loganalyser:loganalyser.token.Token]{\emph{\code{loganalyser.token.Token}}}}

Classe concrète instanciant les token Methode, le format attendu étant l'une des chaines de caractères suivante : GET, HEAD, POST, OPTIONS, CONNECT, TRACE, PUT, DELETE
\index{\_Token\_\_analyse() (loganalyser.token.Method method)}

\begin{fulllineitems}
\phantomsection\label{loganalyser:loganalyser.token.Method._Token__analyse}\pysiglinewithargsret{\bfcode{\_Token\_\_analyse}}{}{}
Analyse la donnee contenue dans le token pour obtenir la sévérité de cette donnee. Non implémenté car non-utile.
\begin{quote}\begin{description}
\item[{Returns}] \leavevmode
Retourne la sévérité de la donnee de ce token

\item[{Return type}] \leavevmode
int

\end{description}\end{quote}

\end{fulllineitems}

\index{\_Token\_\_verifier\_type() (loganalyser.token.Method method)}

\begin{fulllineitems}
\phantomsection\label{loganalyser:loganalyser.token.Method._Token__verifier_type}\pysiglinewithargsret{\bfcode{\_Token\_\_verifier\_type}}{\emph{value}}{}
Vérifie si la donnee peut bien être instanciée sous cette classe de Token.
\begin{quote}\begin{description}
\item[{Parameters}] \leavevmode
\textbf{\texttt{value}} (\emph{\texttt{str}}) -- valeur de création du token

\item[{Returns}] \leavevmode
Retourne la réponse de la vérification

\item[{Return type}] \leavevmode
bool

\end{description}\end{quote}

\end{fulllineitems}


\end{fulllineitems}

\index{URL (class in loganalyser.token)}

\begin{fulllineitems}
\phantomsection\label{loganalyser:loganalyser.token.URL}\pysiglinewithargsret{\strong{class }\code{loganalyser.token.}\bfcode{URL}}{\emph{value}}{}
Bases: {\hyperref[loganalyser:loganalyser.token.Token]{\emph{\code{loganalyser.token.Token}}}}

Classe concrète instanciant les token URL
\index{\_Token\_\_analyse() (loganalyser.token.URL method)}

\begin{fulllineitems}
\phantomsection\label{loganalyser:loganalyser.token.URL._Token__analyse}\pysiglinewithargsret{\bfcode{\_Token\_\_analyse}}{}{}
Analyse la donnee contenue dans le token pour obtenir la sévérité de cette donnee. Non implémenté car non-utile.
\begin{quote}\begin{description}
\item[{Returns}] \leavevmode
Retourne la sévérité de la donnee de ce token

\item[{Return type}] \leavevmode
int

\end{description}\end{quote}

\end{fulllineitems}

\index{\_Token\_\_verifier\_type() (loganalyser.token.URL method)}

\begin{fulllineitems}
\phantomsection\label{loganalyser:loganalyser.token.URL._Token__verifier_type}\pysiglinewithargsret{\bfcode{\_Token\_\_verifier\_type}}{\emph{value}}{}
Vérifie si la donnee peut bien être instanciée sous cette classe de Token.
\begin{quote}\begin{description}
\item[{Parameters}] \leavevmode
\textbf{\texttt{value}} (\emph{\texttt{str}}) -- valeur de création du token

\item[{Returns}] \leavevmode
Retourne la réponse de la vérification

\item[{Return type}] \leavevmode
bool

\end{description}\end{quote}

\end{fulllineitems}


\end{fulllineitems}

\index{Response (class in loganalyser.token)}

\begin{fulllineitems}
\phantomsection\label{loganalyser:loganalyser.token.Response}\pysiglinewithargsret{\strong{class }\code{loganalyser.token.}\bfcode{Response}}{\emph{value}}{}
Bases: {\hyperref[loganalyser:loganalyser.token.Token]{\emph{\code{loganalyser.token.Token}}}}

Classe concrète instanciant les token Réponse, le format attendu étant un entier entre 100 et 599 (compris)
\index{\_Token\_\_analyse() (loganalyser.token.Response method)}

\begin{fulllineitems}
\phantomsection\label{loganalyser:loganalyser.token.Response._Token__analyse}\pysiglinewithargsret{\bfcode{\_Token\_\_analyse}}{}{}
Analyse la donnee contenue dans le token pour obtenir la sévérité de cette donnee. Non implémenté car non-utile.
\begin{quote}\begin{description}
\item[{Returns}] \leavevmode
Retourne la sévérité de la donnee de ce token

\item[{Return type}] \leavevmode
int

\end{description}\end{quote}

\end{fulllineitems}

\index{\_Token\_\_verifier\_type() (loganalyser.token.Response method)}

\begin{fulllineitems}
\phantomsection\label{loganalyser:loganalyser.token.Response._Token__verifier_type}\pysiglinewithargsret{\bfcode{\_Token\_\_verifier\_type}}{\emph{value}}{}
Vérifie si la donnee peut bien être instanciée sous cette classe de Token.
\begin{quote}\begin{description}
\item[{Parameters}] \leavevmode
\textbf{\texttt{value}} (\emph{\texttt{str}}) -- valeur de création du token

\item[{Returns}] \leavevmode
Retourne la réponse de la vérification

\item[{Return type}] \leavevmode
bool

\end{description}\end{quote}

\end{fulllineitems}


\end{fulllineitems}

\index{Byte (class in loganalyser.token)}

\begin{fulllineitems}
\phantomsection\label{loganalyser:loganalyser.token.Byte}\pysiglinewithargsret{\strong{class }\code{loganalyser.token.}\bfcode{Byte}}{\emph{value}}{}
Bases: {\hyperref[loganalyser:loganalyser.token.Token]{\emph{\code{loganalyser.token.Token}}}}

Classe concrète instanciant les token Octet, le fomat attendu étant un entier
\index{\_Token\_\_analyse() (loganalyser.token.Byte method)}

\begin{fulllineitems}
\phantomsection\label{loganalyser:loganalyser.token.Byte._Token__analyse}\pysiglinewithargsret{\bfcode{\_Token\_\_analyse}}{}{}
Analyse la donnee contenue dans le token pour obtenir la sévérité de cette donnee. Non implémenté car non-utile.
\begin{quote}\begin{description}
\item[{Returns}] \leavevmode
Retourne la sévérité de la donnee de ce token

\item[{Return type}] \leavevmode
int

\end{description}\end{quote}

\end{fulllineitems}

\index{\_Token\_\_verifier\_type() (loganalyser.token.Byte method)}

\begin{fulllineitems}
\phantomsection\label{loganalyser:loganalyser.token.Byte._Token__verifier_type}\pysiglinewithargsret{\bfcode{\_Token\_\_verifier\_type}}{\emph{value}}{}
Vérifie si la donnee peut bien être instanciée sous cette classe de Token.
\begin{quote}\begin{description}
\item[{Parameters}] \leavevmode
\textbf{\texttt{value}} (\emph{\texttt{str}}) -- valeur de création du token

\item[{Returns}] \leavevmode
Retourne la réponse de la vérification

\item[{Return type}] \leavevmode
bool

\end{description}\end{quote}

\end{fulllineitems}


\end{fulllineitems}

\index{Referer (class in loganalyser.token)}

\begin{fulllineitems}
\phantomsection\label{loganalyser:loganalyser.token.Referer}\pysiglinewithargsret{\strong{class }\code{loganalyser.token.}\bfcode{Referer}}{\emph{value}}{}
Bases: {\hyperref[loganalyser:loganalyser.token.Token]{\emph{\code{loganalyser.token.Token}}}}

Classe concrète instanciant les token Referer
\index{\_Token\_\_analyse() (loganalyser.token.Referer method)}

\begin{fulllineitems}
\phantomsection\label{loganalyser:loganalyser.token.Referer._Token__analyse}\pysiglinewithargsret{\bfcode{\_Token\_\_analyse}}{}{}
Analyse la donnee contenue dans le token pour obtenir la sévérité de cette donnee. Non implémenté car non-utile.
\begin{quote}\begin{description}
\item[{Returns}] \leavevmode
Retourne la sévérité de la donnee de ce token

\item[{Return type}] \leavevmode
int

\end{description}\end{quote}

\end{fulllineitems}

\index{\_Token\_\_verifier\_type() (loganalyser.token.Referer method)}

\begin{fulllineitems}
\phantomsection\label{loganalyser:loganalyser.token.Referer._Token__verifier_type}\pysiglinewithargsret{\bfcode{\_Token\_\_verifier\_type}}{\emph{value}}{}
Vérifie si la donnee peut bien être instanciée sous cette classe de Token.
\begin{quote}\begin{description}
\item[{Parameters}] \leavevmode
\textbf{\texttt{value}} (\emph{\texttt{str}}) -- valeur de création du token

\item[{Returns}] \leavevmode
Retourne la réponse de la vérification

\item[{Return type}] \leavevmode
bool

\end{description}\end{quote}

\end{fulllineitems}


\end{fulllineitems}



\subsection{Outils module}
\label{loganalyser:outils-module}\label{loganalyser:module-loganalyser.outils}\index{loganalyser.outils (module)}
Sert à l'implémentation de notre classe Dictionnaire, qui étend la classe dict de Python, et y ajoute les opérations
qui nous sont utiles sur les dictionnaires.
\index{Dictionary (class in loganalyser.outils)}

\begin{fulllineitems}
\phantomsection\label{loganalyser:loganalyser.outils.Dictionary}\pysiglinewithargsret{\strong{class }\code{loganalyser.outils.}\bfcode{Dictionary}}{\emph{keylist}}{}
Bases: \code{dict}

Extension de la classe dictionnaire. Cette classe possède comme attributs supplémentaires:
- La liste des clefs du dictionnaire
\index{\_\_init\_\_() (loganalyser.outils.Dictionary method)}

\begin{fulllineitems}
\phantomsection\label{loganalyser:loganalyser.outils.Dictionary.__init__}\pysiglinewithargsret{\bfcode{\_\_init\_\_}}{\emph{keylist}}{}
Constructeur de classe. Un dictionnaire est initialisé vide à partir de la liste des clefs
\begin{quote}\begin{description}
\item[{Parameters}] \leavevmode
\textbf{\texttt{keylist}} (\emph{\texttt{list}}) -- Liste des clefs du dictionnaire.

\end{description}\end{quote}

\end{fulllineitems}

\index{keys() (loganalyser.outils.Dictionary method)}

\begin{fulllineitems}
\phantomsection\label{loganalyser:loganalyser.outils.Dictionary.keys}\pysiglinewithargsret{\bfcode{keys}}{}{}
Retourne les clefs du dictionnaire.
\begin{quote}\begin{description}
\item[{Returns}] \leavevmode
Liste contenant les clefs du dictionnaire.

\item[{Return type}] \leavevmode
list

\end{description}\end{quote}

\end{fulllineitems}

\index{addentry() (loganalyser.outils.Dictionary method)}

\begin{fulllineitems}
\phantomsection\label{loganalyser:loganalyser.outils.Dictionary.addentry}\pysiglinewithargsret{\bfcode{addentry}}{\emph{entry}}{}
Ajoute au dicitonnaire une nouvelle valeur dans chacunes de ses clefs à partir d'une liste.
\begin{quote}\begin{description}
\item[{Parameters}] \leavevmode
\textbf{\texttt{entry}} (\emph{\texttt{list}}) -- Liste contenant les valeurs pour chacune des clefs

\end{description}\end{quote}

\end{fulllineitems}

\index{getentry() (loganalyser.outils.Dictionary method)}

\begin{fulllineitems}
\phantomsection\label{loganalyser:loganalyser.outils.Dictionary.getentry}\pysiglinewithargsret{\bfcode{getentry}}{\emph{entrynumber}}{}
Retourne la liste contenant les valeurs de chaques clefs pour un index donné.
\begin{quote}\begin{description}
\item[{Parameters}] \leavevmode
\textbf{\texttt{entrynumber}} (\emph{\texttt{int}}) -- index de l'entrée.

\item[{Returns}] \leavevmode
Liste contenant les valeurs de chaques clefs pour le même index.

\item[{Return type}] \leavevmode
list

\end{description}\end{quote}

\end{fulllineitems}

\index{itemtoentrynumbers() (loganalyser.outils.Dictionary method)}

\begin{fulllineitems}
\phantomsection\label{loganalyser:loganalyser.outils.Dictionary.itemtoentrynumbers}\pysiglinewithargsret{\bfcode{itemtoentrynumbers}}{\emph{item}}{}
Retourne l'index d'une valeur dans le dictionnaire.
\begin{quote}\begin{description}
\item[{Parameters}] \leavevmode
\textbf{\texttt{item}} (\emph{\texttt{str}}) -- valeur recherchée.

\item[{Returns}] \leavevmode
Liste contenant les index associés à la valeur d'entrée.

\item[{Return type}] \leavevmode
list

\end{description}\end{quote}

\end{fulllineitems}


\end{fulllineitems}



\chapter{Crédits}
\label{index:credits}
Projet conduit par Jeremy Venin, Clément Aubry, Antoine Tadros, Anatole Lefort dans le cadre du programme
d'enseignement de Telecom SudParis.


\renewcommand{\indexname}{Python Module Index}
\begin{theindex}
\def\bigletter#1{{\Large\sffamily#1}\nopagebreak\vspace{1mm}}
\bigletter{l}
\item {\texttt{loganalyser.diagnostique}}, \pageref{loganalyser:module-loganalyser.diagnostique}
\item {\texttt{loganalyser.fichier}}, \pageref{loganalyser:module-loganalyser.fichier}
\item {\texttt{loganalyser.outils}}, \pageref{loganalyser:module-loganalyser.outils}
\item {\texttt{loganalyser.token}}, \pageref{loganalyser:module-loganalyser.token}
\end{theindex}

\renewcommand{\indexname}{Index}
\printindex
\end{document}
